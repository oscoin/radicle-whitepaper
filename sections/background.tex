\section{Background}

To motivate this work, we must first familiarize ourselves with the conditions
in which open-source software is developed, as well as some of the core
technologies and ideas which inspired this work.

\paragraph{The Open-Source Movement.} Software whose source code is publicly
available is called \emph{open-source}. Most of the software we use on a daily
basis relies on free and public code. In an increasingly digitalized society,
these open-source projects have become the foundation of digital infrastructure
underpinning many of our societal goods and services.

Over the last few years, with the emergence of software hosting sites like
GitHub and community sites like Stack Overflow, the open-source paradigm became
the focal point of software development, resulting in the development of
numerous high quality projects wwhich are openly available for anyone to use.
This phenomenon helped companies (a) to reduce lead times and bring products to
the market faster and (b) to recruit talent from a new pool of technologists,
educated on the basis of open-source software.

Today, many open-source projects are started by individuals or small groups of
people solving a personally, socially, or technically relevant problem. By
analysing people's motivations for participating in open-source software
development the following themes emerge: (a) pride in one's work, (b)
reputation, (c) learning, (d) responsibility for something they believe in,
i.e.\ \emph{the project got attention, I have to maintain it}, (e) being part
of a community and (f) financial compensation.

While most open-source software projects start for the reasons above, the ones
that gain momentum require significant resources in time and money to maintain.
This creates a fundamental problem: projects created in the spare time of a
developer are becoming critical public infrastructure. As their projects
gain popularity, developers experience stress and exhaustion trying to keep up
with the requests of the community during their free time, often abandoning
their own projects or burning out under the increased responsibility. Left
unchecked, this leads to a tragedy of the commons~\cite{tragedy-commons} which
is at the core of the problem \oscoin{} attempts to address: financial
sustainability and transparency in the development of open-source software.

% TODO: Talk about the effect on code too!

\paragraph{Version control systems.} At the core of open-source collaboration
is the version control system, or VCS---a system for tracking changes to source
code over time. Version control systems have long been used to manage
open-source software projects and are an essential part of how open-source is
developed, allowing large groups of people to efficiently coordinate around a
project and track contributions individually. In the last two decades, a new
kind of \emph{distributed} VCS has emerged, known as a DVCS, the most popular
of which is \texttt{git}~\cite{git}. But despite the distributed nature
and popularity of \texttt{git}, we are experiencing a new form of
centralization around platforms such as GitHub and GitLab.  What used to be an
open and decentralized process\footnote{Before sites likes GitHub, the majority
of open-source contributions were sent as patches to maintainers on mailing
lists. This is still how the Linux kernel project operates} is now controlled
by a few major players. This is largely due to the convenience of these
platforms, yet few consider the implications of entrusting the world's code to
a commercial organization and relying on a centralized platform which has been
shown to be a single point of failure on more than one occasion.

\paragraph{Blockchain technology.} In 2009, with the publication of the Bitcoin
paper~\cite{bitcoin}, blockchain technology was introduced. Blockchain enables
a peer-to-peer network of participants to agree on the state of a global
\emph{transaction ledger} in the \emph{permissionless} setting. In this
setting, participation in the agreement protocol of the network---also known as
\emph{consensus}---is open and public.

However, what was perhaps the most important discovery took some time to reach
collective consciousness. This was the use of \emph{economic incentives} at the
protocol level. In this model, network operators, or \emph{miners} in the
case of Bitcoin, are incentivized to keep operating the network and to
collaborate with each other, because disagreement has an economic risk.

This inspired a resurgence in protocol thinking and innovation in open network
design, which is the foundation on which we are basing our solution to the
open-source sustainability problem.

Beyond the pressing issues in financial sustainability, open-source projects
organized in a decentralized manner, which is the case for a majority of
blockchain-based projects such as Ethereum and Bitcoin, are lacking the tools
necessary for coordination and decision making around protocol changes.

And although decentralized systems are computationally inefficient, they are
often more resilient, sustainable and available than their centralized
counterparts, properties which are important when designing public
infrastructure.

\pagebreak
