\section{Seed Protocol Semantics}
\label{sec:protocol-semantics}

% TODO: Distinguish operations from messages, make sure messages aren't signed.
% TODO: Make sure all operations, ex. around issues are using a message token.

With the core architecture out of the way, we can now describe the \oscoin{}
seed protocol in terms of $\fold$ and $\state$. Since the protocol is amendable
due to our definition of $\fold$, we define a \emph{seed} protocol, which has
the ability to evolve based on its user's needs.

The rules according to which transactions on the \oscoin{} network are
validated comprise a \emph{protocol}. This protocol has a well defined semantics
which we shall describe in this section.

\subsection{Overview} The \oscoin{} protocol is composed of interrelated
\emph{objects} which form a hierarchical graph as seen in Figure
\ref{object-relationships}, and \emph{operations}, which act on these objects.

\medskip

\begin{fig}
\centering
\begin{tikzpicture}[sibling distance=8em]
    \node {\emph{root}}
        child { node {\emph{tokens}} }
        child { node {\emph{accounts}}
            child { node {\emph{keys}} }
            child { node {\emph{repositories}}
                child { node {\emph{patches}} }
                child { node {\emph{proposals}}
                    child { node {\emph{patchset}} }
                    child { node {\emph{votes}} } } } };
\end{tikzpicture}
\bigskip
\captionof{figure}{Object hierarchy in the \oscoin{} protocol.
\label{object-relationships}}
\end{fig}

\subsection{The Root}
In the beginning, we have only one chain, known as the \emph{root}. In its simplest
form, the genesis of the root chain would be defined as an empty state $\varnothing$
and a fold function $f$ which allows only one operation called $branch$, which takes
an identifier, a fold function of its own and creates a branch with the given
fold function as $\fold$.

% TODO: This is no longer true. We have eval by default.

\subsection{Operations and State}
\label{operations-and-state}

% TODO: Call operation messages 'commands'. Then refer to all future
% operations as commands. Move the author, nonce etc. to the command. Make
% operations as a sequence of commands.

% TODO: Better word for Contexts?
% TODO: Issue resolution. Voicing. Voicing can be used to implement voting mechanisms.
% TODO: Merge accounts (A) and orgs (O).
% TODO: Change apply function to *fold* function.

Operations are signed messages constructed by participants in the protocol,
which trigger a state transition in the network. An operation $\op$ takes
the form:
\[
    \op \equiv \tuple{\op_i, \op_m, \op_p, \op_n}_{\sigma}
\]
where $\op_i$ is the identity of the sender of the operation, $\op_m$ is the
message, $\op_p$ is a reference to a parent operation, such that $\op_p \prec
\op$. Finally, $\op_n \in \mathbb{N}$ taken together with $\op_i$ forms a
globally unique operation identifier, such that there can be at most one
operation with a given $\tuple{\op_i, \op_n}$ tuple. If $\mathcal{T}$ is the
set of all operations, then $\mathcal{T}_v \subset \mathcal{T}$ is the set of all \emph{valid}
operations.

An operation $t \in \mathcal{T}_v$ applied to the current state of the protocol $\mathcal{S}_e$
at epoch $e$ can be formulated as:
\[
    \mathcal{S}_{e+1} \equiv \apply(\mathcal{S}_e, t)
\]
where $\apply$ is the state-transition function defined by:
\[
    \apply(\State, t) \equiv \begin{cases}
        \apply_{s}(\State, t) & \text{if} \quad t \in \mathcal{T}_v \\
        \State                & \text{otherwise}
    \end{cases}
\]
and $\apply_s$ is a sub-function of $\apply$ which applies valid operations to
$\State$.  We can describe the current
state of the network $\mathcal{S}_e$ as a sequence of operations $\mathcal{L} =
\op_1 \dotso \op_e$, (the ``ledger'') applied recursively to an initial empty
state $\varnothing$:
\[
    \mathcal{S}_e \equiv \apply(\dotso \apply(\apply(\apply(\varnothing,
    \op_1), \op_2), \op_3) \dotso \op_e)
\]
The state $\mathcal{S}$ of the protocol is defined as:
\[
    \mathcal{S} \equiv \tuple{\mathcal{A}, \mathcal{O}, \mathcal{D}, \mathcal{C}, \mathcal{L}} \\
\]
where $\mathcal{A}$ is the set of accounts (\S \ref{accounts}), $\mathcal{O}$
is the set of organizations known to the protocol (\S \ref{orgs}),
$\mathcal{D}$ is the set of dependencies between projects (\S
\ref{dependencies}), $\mathcal{C}$ is the set of contexts onto which patches
can be applied (\S \ref{contexts}) and $\mathcal{L}$ is the sequence of
operations which have been applied to $\mathcal{S}$. It is thus possible to
reconstruct $\mathcal{S}$ from $\mathcal{L}$ via the $\apply$ function, as
seen above. In the remainder of the section, we shall use the word
\emph{operation} to mean a valid message that has been applied to the state
$\State$ and recorded in a ledger $\Ledger$, and \emph{message} to mean an
object sent between participants of the protocol to communicate.

\subsection{Patches}
\label{patches}

The fundamental product or \emph{artifact} of the \oscoin{} network is code, or
source code. Our protocol defines the \emph{patch} primitive as the atomic unit
of code. A patch is a set of metadata and code changes, or \emph{changeset},
that can be applied to an existing body of code, or \emph{context}, to modify
it. We define a patch $P$ as the tuple $\tuple{P_i, P_m, P_f}$, where $P_i$ is
the identity of the author of the patch, $P_m$ is the patch metadata and $P_f$
is the changeset. A set of patches in no particular order is called a
\emph{patchset}.

Patches can be composed sequentially, taking an initial context $A$ into
a modified context $B$. The empty context is defined as $\varnothing$, such
that there exists a mapping $P_f \varnothing \mapsto P_f$. Let $f \prec g
\prec h$ be an ordered sequence of patches, then the formulation:
\[
    h \cdot g \cdot f : A \to B
\]
is the in-order composition of patches $f$, $g$ and $h$, taking an initial
context from $A$ to $B$. In \oscoin{}, typical examples of contexts include
code repositories, files and branches.

% TODO: Repository section. A repository is a context C, initially {} with an
% identifier (to which patches can refer), that patches can be applied to,
% and issues can refer to.

% TODO: In git, a repository contains: a set of commit objects, a set of
% references called 'heads', to commit objects (ex. branches).
% A project always has one commit object with no parents, this is the first
% commit.

% TODO: Storing patch metadata (instead of hash, and not the changeset) is for
% 'lazy-repositories' - you can download them and read the history without
% downloading the files.

% TODO: The sequence of patches is called the 'inventory'. Doesn't seem to have
% parent pointers in darcs!

% TODO: There's a "set of patches" and a "sequence of patches". They are both
% useful views on the dataset.

% "any permutation of a patch sequence allowed by Darcs leads to the same exact repository (if not, we have a bug)" ?!
% TODO: Therefore there is a set of valid orders which lead to the same repository!

\subsubsection{The \textsc{patch} message}
\label{patch-op}

To exchange patches amongst protocol participants, we define the
$\tx{patch}{p}$ message, which allows a patch or patchset $p$ to be broadcast
to all participants.

\subsubsection{The \textsc{record} message}
\label{record-op}

In order for a patch or patchset to be applied to some context $r_c \in
\mathcal{C}$ , we define a message $\tx{record}{r_p, r_c}$, which takes a
patchset $r_p$ and applies it to a context $r_c$.

\subsubsection{The \textsc{fork} message}
\label{fork-op}

Given an existing context $k_c$, the $\tx{fork}{k_c, k_{c'}}$ message derives
a new context $k_{c'}$ which shares all existing patches with $k_c$. From that
point on, $k_c$ and $k_{c'}$ are allowed to diverge.

The \textsc{fork} message can also be used to create entirely new contexts, by
passing the empty context $\mathcal{C}_\varnothing$ as $k_c$.

\subsection{Contexts}
\todo{}
% TODO: (Account, {Repo, File, ...})

% TODO: Concept of ownership!?

% TODO: Are patches tied to orgs/repos/branches, or free-floating, and the
% PATCH operation is what is tied to them?


% TODO: Contexts form a free monoid?
% TODO: Reference patch theory.

\subsection{Issues}
\label{issues}

On the \oscoin{} network, collaboration on code takes place through \emph{issues};
and since project governance is specified in code---through the use of
\emph{smart contracts}---organizational decisions are also made through issues.

% TODO: Remove reference to smart contracts!

An issue $I$ is described at epoch $e$ by the tuple:
\[
    \big<\tuple{I_a, I_t, I_b, I_o, I_r, I_P}_{\sigma}, I_s, I_V \big>_e
\]
where $I_a$ is the issue author, $I_t$ is the title or subject of the issue,
$I_b$ is the description in plain text of the issue, $I_o$ is a list of
operations to be applied when $I_s$ changes, $I_r$ is the issue resolution
function, $I_P$ is the patchset attached to the issue, $\sigma$ is the
signature of $I_a$, $I_s \in \{open, closed, accepted, rejected\}$ is the
current state of the issue and $I_V$ is the set of votes on the issue. The
initial value of $I_s$, $I_{s_0} = open$, the initial value of $I_V$, $I_{V_0}
= \varnothing$ and the initial value of $I_P$, $I_{P_0} = \varnothing$.  The list
of valid state transitions between any two states $I_s$ and $I_{s'}$ are as
follows:

\begin{fig}
    \centering
    \captionof{table}{Valid state transitions.\label{issues-valid-transitions}}
    \begin{tabular}{@{}rcl@{}}
        \toprule
        $I_s$      & $\to$ & $I_{s'}$ \\
        \midrule
        $open$     & $\to$ & $accepted$ \\
        $open$     & $\to$ & $rejected$ \\
        $open$     & $\to$ & $closed$ \\
        $closed$   & $\to$ & $open$ \\
        \bottomrule
    \end{tabular}
\end{fig}

The patchset $I_P$ attached to an issue represents a set of proposed changes to
some project, while the subject $I_t$ and body $I_b$ of the issue provide a
description of the changes contained in $I_P$. Issue \emph{resolution} is the
process of voting on an issue with the aim to move to an \emph{accepted} or
\emph{rejected} state.

Issues can be voted on with the \textsc{voice} operation. When an issue receives
a new vote, it is added to $I_V$. A vote $v$ is represented by the tuple
$\tuple{v_i, v_a, v_{\omega}, v_e}_{\sigma}$ where $v_a$ is the voter,
$v_i$ is the issue being voted on, $v_{\omega} \in \{accept, reject\}$ and
$v_e$ is the epoch $e$ at which the vote is valid.

When a certain threshold of votes is reached, an issue transitions to either an
\emph{accepted} or \emph{rejected} state. Given an open issue $i$, the rules of
issue resolution are defined by the function $i_r : I \to I$, applied to the
issue $i$ for every vote added to $i_V$.

\subsubsection{Amendments}

An issue $I$ where $I_s = open$ can be amended with the \textsc{amend}
operation, or $\amend$. Only $I_t$, $I_b$, $I_o$, $I_r$ and $I_P$ can be
amended.  Amending an issue creates a new empty set of votes $V'$, ensuring two
versions of a given issue never share a set of votes. Formally, amendment is
defined as:
\begin{align*}
    I \amend{} \langle I_a, t', b', o', r', P' \rangle_{\sigma} \equiv
    \big<\langle I_a, t', b', o', r', P' \rangle_{\sigma}, I_s, \varnothing
    \big>
\end{align*}
where $I_s = open$.

% TODO: The amendment should include the issue it is amending.

\subsubsection{Accepted Issues} When an issue has been accepted by a majority
of votes, the issue transitions permanently into an \emph{accepted} state. The
steps taken by the protocol are as follows:

\begin{enumerate}
    \item The issue's state $I_s$ is set to \emph{accepted}.
    \item The issue is \emph{frozen}, such that no further amendments or state
        changes are possible.
    \item The list of operations $I_o$ belonging to the issue are executed by
        the protocol.
    \item The issue's \emph{patchset} $I_P$ is permanently added to the code
        project it pertains to. Note that a patchset may contain individual
        patches pertaining to different projects, in which case the patches are
        applied individually to their respective projects.
\end{enumerate}

\subsubsection{Rejected issues} When an issue is rejected,
\begin{enumerate}
    \item The issue transitions to a \emph{rejected} state.
    \item The issue is frozen so that no further amendments or state
        transitions are possible.
    \item The list of operations $I_o$ belonging to the issue are executed by
        the protocol.
\end{enumerate}

\subsubsection{Closed issues} When an issue is closed, its state changes to
\emph{closed} until it is re-opened. No operations from $I_o$ are run, since
an issue can be opened and closed many times. Only the author $I_a$ of an issue
can close it.

\subsection{Value}

To enable the transfer of value in the network, the protocol defines a scarce
currency we shall refer to in the remainder of this paper as~\coin{}, or
\oscoin{}.

\subsubsection{Supply}

The supply of \coin{} is subject to an inflation $e_{\iota} \in \mathbb{N}$,
carried out by the protocol every epoch $e$, and determined by a logarithmic
function $f : e \to e_{\iota}$, such that $\lim_{e \to \infty} f(e) = 0$.

\subsubsection{Accounts}
\label{accounts}

Currency is held in \emph{accounts} which can be unlocked by the signature of
the account holder. Accounts have addresses which are used to send and receive
\coin{}. The set of all accounts is known as $\mathcal{A}$.

\subsubsection{Transfer}

Value can be transfered from one account to another with the \textsc{send}
message, formally defined as $\tx{send}{a_{s}, a_{r}, n}$, where $a_{s}$ and
$a_{r}$ are the \emph{sender} and \emph{recipient} addresses between which the
value should be transfered and $n$ is the value to transfer.  To be valid, a
\textsc{send} operation must be signed by the owner of $a_{s}$.

\subsubsection{Bonding}

Value can be locked in the system via an operation called \emph{bonding}.
This operation turns \emph{liquid} value into \emph{bonded} value, preventing
them from being moved for a certain amount of time, and can be used to perform
security deposits or other forms of commitment which require a collateral or
pledge. Bonding and unbonding operations are performed with the
\textsc{bond} and \textsc{unbond} messages defined as:
\[
    \tx{bond}{a, a_b, n} \qquad \tx{unbond}{a, a_b, n}
    \medskip
\]
where $a$ is the address from which to withdraw the value for bonding, $a_b$ is
the address where the value is to be bonded and $n$ is the value to bond. When
the \textsc{unbond} operation is used, an unbonding period $e_u$ is started,
measured in epochs. Once $e_{u}$ epochs have passed, the value is withdrawn
from the bonding address $a_b$ and credited back to $a$.

\subsection{Organizations}
\label{orgs}

An organization $O$ is described by the tuple:
\[
    \tuple{O_{id}, O_{a}, O_R, O_M}
\]
where $O_{id}$ is the organization's identifier, $O_a$ is its account, $O_R$ is
the set of repositories under $O$ and $O_M$ is the set of members belonging to
the organization. We can relate patches (\ref{patches}) and issues
(\ref{issues}) to repositories with the equations:
\begin{align*}
    O_R & \equiv \{R_1, R_2, \dotsc, R_n\}         \\
    R   & \equiv \tuple{R_{id}, R_I, R_P}         \\
    R_I & \equiv \{I_1, I_2, \dotsc, I_n\}         \\
    R_P & \equiv \tuple{P_1, P_2, \dotsc, P_n}
\end{align*}
where $R_{id}$ is the repository's identifier, $R_I$ is the set of issues
related to $R$, and $R_P$ is the sequence of patches composing its codebase.

The set $\mathcal{O} = \{O_1, O_2, \dotsc, O_n\}$ is the set of all organizations
known to the protocol.  Let $o$ be an organization, then $\tx{register}{o}$ is
an operation that registers $o$ and makes it known to the protocol such that:
\[
    \apply_o(\mathcal{O}, \tx{register}{o})
    \equiv \mathcal{O} \cup \{o\}
\]
where $\apply_o$ is a sub-function of $\apply$, and defines
state-transitions in $\mathcal{O}$.

\subsection{Dependencies}
\label{dependencies}

A project $a$ depends on a project $b$ at epoch $e$ if it references $b$
or parts of $b$. Formally, we represent this as:
\[
    D \equiv \tuple{D_a, D_b, D_e} \qquad \text{or} \qquad a \dep b
\]

% TODO: dependencies with \overrightarrow{AB}, like vectors.

Let $R^*$ be the set $O_{1_R} \cup O_{2_R} \cup \cdots \cup
O_{n_R}$ of all projects across all organizations, and $\mathcal{D}_e$ be the
set of all dependencies at epoch $e$, then:
\[
    \mathcal{D}_e \equiv \{(a, b) : a \in R^*
    \land b \in R^*
    \land a \dep b \}
\]

Dependencies are recorded with the
\[
    \tx{depend}{D_a, D_b, D_e, D_{e'}}
\]
operation, which records a dependency between $a$ and $b$ for each epoch $x$
where $e \leqslant x \leqslant e'$.

\subsection{Issues \& Proposals}

Collaboration around open-source in \oscoin{} can be seen as a three step
process.  First, an issue is identified and discussed, second, a proposal is
made to resolve this issue, and finally the proposal is accepted by a
maintainer and the changes proposed are merged into the codebase.

It's important to note that there can be at most three unique parties involved
in this collaborative effort: the author of the issue, the author of the
proposal, and the maintainer of the repository to which the issue and proposal
pertain.

Therefore, it's advantageous to decouple these three phases as much as
possible, in order to reduce friction between the parties, and reduce
incidental coordination costs to the process -- in other words, we try to
postpone agreement to the latest phase, when it is really needed, to allow as
much work to be done independently.

Practically, this means that issues and proposals can be created independent of
any maintainer, and only when the time comes to accept or reject a proposal, is
the maintainer called to action. This is in contrast to more centralized
systems that gate the issue creation process by having a single, canonical list
of issues per project. In \oscoin{}, anyone can maintain a personal issue or
proposal list for a project, without consent from the owner. The goal here is
to reduce friction between the different parties, and embrace the decentralized
nature of the protocol.

We shall now formally define issues and proposals, and explore a few different
examples of the system in practice.

\subsubsection{Issues}

An issue in essence is a state modification function, or ``smart contract'',
attached to a repository. An issue $I$ is define as
\[
    I \equiv \tuple{I_{id}, I_s, I_f, I_r, I_m}_{\sigma},
\]
where $I_{id}$ is the \blake{} hash of the issue, $I_s \in \{open, closed,
resolved\}$ is the current state of the issue, $I_f : (I, \state) \to (I,
\state)$ is a function that is called when $I_s$ changes, and can modify the
state $C_\state$ of the chain. $I_m$ is the hash of some off-chain metadata
related to the issue (title, body, tags, etc.), and $I_r$ is the \blake{} hash
of the repository the issue pertains to.

The function $I_f$ can be used to implement incentivization schemes such as
bounties (\S \ref{bounties}), [...] etc.

The metadata hash $I_m$ is used to relate the issue with for example a subject,
description or set of tags that may be useful at the application layer, but
isn't crucial to the functioning of the chain.

\subsubsection{Proposals}

Issues on their own have very little to offer -- it's with proposals that they
become powerful and that their purpose becomes clear. A proposal, defined as
\[
    P \equiv \tuple{P_p, P_I, P_s, P_r, P_V}_{\sigma}
\]
is a set of patches $P_p$ relating to zero or more issues $P_I$, that proposes
a change to some repository $P_r$. The relationship between a proposal and
an issue is as follows: if a proposal $P$ is accepted as the solution to
some issue $i$, then the contract $i_f$ will be run with $P$ as input. Thus,
$i_f$ is able to do such things as merge code from $P$ and unlock bounties
for the author of $P$.

% \subsubsection{Operations on issues and proposals}

% On issues:
% amend   : Hashed Meta -> Script -> Issue s -> Issue s
% close   : Issue Open -> Issue Closed
% open    : Issue Closed -> Issue Open
% resolve : Issue Open -> Maybe Proposal -> Issue Resolved

% On proposals:
% vote    : Vote -> Proposal Open -> Proposal (Open | Accepted | Rejected)
% amend   : Set (Hashed Issue) -> Set Patch -> Proposal Open -> Proposal Open
% close   : Proposal Open -> Proposal Closed
% open    : Proposal Closed -> Proposal Open

