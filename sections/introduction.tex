\section{Introduction}

\begin{epigraph}{The Open-Source Everything Manifesto}
    \noindent It is in this light that we must recognize that only a restoration of
    open-source culture, and all that enables across the full spectrum of
    open-source possibilities, can allow humanity to harness the distributed
    intelligence of the collective and create the equivalent of heaven on Earth
    -- in other words, a world that works for all.
\end{epigraph}

\noindent \oscoin{} is a protocol and token that attempts to provide a decentralized
hosting solution for open-source code with the means to collaborate on, govern
and fund open-source software projects sustainably, with no central authority
in control.

\subsection{Core Components}

\oscoin{} introduces a few components that, taken together, constitute the core
of the proposed solution.

\begin{description}
    \item[Registry] The registry, formally $\mathcal{R}$, is a
        decentralized service which provides a canonical record of projects and
        organizations known to the network.
    \item[Treasury] The treasury, formally $\mathcal{T}$, is a
        decentralized service with the purpose of funding open-source projects
        of value on the network.
    \item[Network] A decentralized code hosting network, formally
        $\mathcal{N}$, which hosts all the source code known to the network
        in a way that is available and censor-proof.
    \item[Dependency Graph] The dependency graph $\mathcal{D}$, is a global
        graph of all dependencies known to the network, whether these
        dependencies are hosted on the network or hosted externally.
    \item[Oscoin] The \oscoin{} token is a cryptographic token used for value
        exchange in the network.
\end{description}

\subsection{Protocol Overview}

\begin{itemize}[itemsep=0pt]
    \item The \oscoin{} network is a decentralized code hosting network
        composed of different kinds of participants, including
        \emph{maintainers}, \emph{contributors} and \emph{operators}.
    \item Maintainers and contributors collaborate around software projects
        organized in code repositories hosted by the network.
    \item Through the registry $\mathcal{R}$, participants discover, support
        and join open-source projects.
    \item Funds available for distribution are first sent to the treasury
        $\mathcal{T}$, before being distributed to chosen organizations.
    \item Token holders are able to pledge a certain amount of tokens towards
        an organization. This has the effect of signaling the treasury that a
        given organization has value to them, influencing the distribution of
        funds.
    \item Organizations are able to use the funds allocated to them for
        whichever purpose they see fit. \emph{Issues} and \emph{smart
        contracts} are used as a means for smart distribution of tokens from
        within an organization, to its constituent members.
    \item Code is attached to issues in the form of \emph{patches}. Issues act
        as the epicenter of change and collaboration around a project.
\end{itemize}
\pagebreak
