\section{Introduction}
\label{s:introduction}

% Replicated state machines. Examples (Paxos, blockchains, etc.). It's purpose
% and constraints

% Variety of applications. Is there a unifying abstraction? Needs to be
% deterministic, but aside from that, a DSL-language.

% Mention radicle.

% Allows for updates that benefit from the same consistency guarantees as the
% operations themselves. Helps solve the problem of software updates in Paxos
% etc, and a more powerful form of self-amendment in blockchains. [Note that
% perfomance gains in the compiler/interpreter must happen differently, though
% separating changes that can be semantic from those that are not helps.]

% Effect system. Separating out impurity. In the case of blockchains, clients
% are *also* programmable, without sacrificing consensus.

% outline.
% Section 2) Some examples: an upgradable key-value store. A version of nomic.
% Section 3) Formal description of language.

Replicated state machines are a widely used paradigm to program fault-tolerant
systems. The paradigm involves deploying a deterministic state machine across
multiple nodes (servers); these nodes can respond to client requests, and by
agreeing on the order of these requests (consensus), ensure agreement on their
state and output. If some fraction of nodes is unavailable (or in the looser
requirement of blockchains, if they are malicious), the overall system can
still function correctly.

The service these systems replicate may be key-value stores, file-systems,
append-only logs, account balances, etc. Each of these
services is often re-implemented anew, leading to substantial development
costs, as well as to subtle bugs in the interim state of the system during the
inevitable software upgrades. In this paper, we describe a language, \rad, for
defining the behaviour of replicated state machines (RSMs) independently of the
underlying consensus. The language is designed to easily allow definition of
new domain-specific languages (DSLs) for the services provided by the system.
Each such DSL is the definition of an RSM. The expressions of the DSL
are the RSMs inputs; the value such an expression evaluates to its outputs; and
the changes in the environment its state changes. If the RSM we are interested
in is a ledger of accounts, expressions or inputs may be transfers, the
state may be the balance and ownership of accounts, and outputs may be the new
balances of affected accounts or an error message if the transfers are not
allowed (due to insufficient funds or incorrect permissions).

Additionally, using the same mechanism for DSL-definition (namely, a
\textit{reflective tower}), we provide a way for upgrading the DSL itself
\textit{with the same guarantees of agreement} between nodes as the underlying
consensus, reducing the coordination difficulty of an upgrade. DSL
\textit{re}-definitions are themselves just another input to the RSM; thus
nodes will agree on the ordering of the upgrade with respect to other inputs,
and will not go out of sync as a consequence of update.

While (crucially for the purpose of ensuring consensus) the core of language is
deterministic, \rad also possesses an additional set of primitive operations
that allow side-effect. We show how a publish-subscribe model for side-effects
(or in the formalism of \cite{Cartwright1994}, a effect-handling central which
is never provided continuations) allows responding to outputs of the state
machine or state changes in a non-deterministic way, without endangering the
determism of the state machine itself. This in turn makes the separation
between reads and writes correct by construction.

\rad has been developed in the context of a broader effort to create a
community-owned platform for open-source development (\oscoin{}), which
includes a consensus algorithm and a networking component. The \oscoin{}
platform allows users and communities to create permissioned and permissionless
RSMs with their own semantics, be it to maintain decentralized
version-controled code, issues, pull requests, or decision-making. \rad is
oriented towards making that process and simple and clear as possible. We do
not in this paper further discuss the broader \oscoin{} system.

In the rest of the paper, we show some example applications built with
\rad--first, an upgradable key-value store, and then more ambitiously, a game
of Nomic (Section \ref{s:examples}). We then describe the language in more
detail (Section \ref{s:language}), and compare it to other technologies.
