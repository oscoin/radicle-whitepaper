\section{Sample Programs and Chains}
\label{s:examples}

In this section, we develop a better sense for the language and its
applications by considering sample programs.

\subsection{Self-amending key-value store}

In Section \ref{s:language}, we defined a simple key-value store language. It
was not, however, an \emph{amendable} one---once defined, it was impossible to
change the semantics of the running system. For short-lived chains with a narrow
purpose, this might be satisfactory. However, for long-lived chains, participants'
ideas as to the purpose of the chain may change over time. Forking to a new chain is
an option, but this is not ideal for two reasons:
\begin{itemize}
  \item Consensus on the purpose and semantics of the new chain must be achieved
    ``off chain''.
  \item Participants must agree on the process and logistics of migrating to a
    new chain, how to decide from which block this takes place, etc.
\end{itemize}
In radicle this can all take place on-chain, by updating the eval function.
\input{out/kv1.rad-tex}
Note that in this case, the new evaluation function that is instantiated as a
result of an \texttt{update} command, is the result of evaluating an
expression with the original \texttt{eval} function, because of the hyperstatic
environment. Thus, in this case, no tower of interpreters is formed. Instead,
we are swapping out one eval for another.

\subsection{Currency}

In this section we define a currency. This example demonstrates both higher
order functions and data-hiding. Indeed the \texttt{create-currency} function
defines all the state needed for the operation of the currency but then only
returns the relevant evaluation function, making the internals of the currency
unavailable to the caller. Furthermore, by returning a potential evaluation
function (rather than setting it directly), this function may be used as a
sub-behaviour of a more featurefull RSM.
\input{out/currency.rad-tex}
After loading this code, we can imagine the following \rad{} session:
\bigskip
\begin{Verbatim}[fontsize=\small]
> (write-ref eval-ref (create-currency))
=> ()
> (new-account "alice")
=> :ok
> (new-account "bob")
=> :ok
> (transfer "alice" "bob" 3)
=> :ok
> (balance "alice")
=> 7
\end{Verbatim}

An obvious problem with this chain is that anyone can submit the command
\texttt{(transfer "bob" "malicious123" n)}. To mitigate this one should use a
function \texttt{signature-valid?} which checks that a cryptographically signed
message has been signed by a specific public key. One can then create a token
currency, with the transfer of tokens only taking place if the command is
appropriately signed. Such a function could be written from scratch in \rad{},
or provided as a built-in function.

\subsection{Updatable state-machine}

Some chains will operate simple state machines that are best described as a
radicle function \texttt{transition} which takes the current state and an input
and returns a new state. For example we might want to maintain a number:
\begin{lstlisting}
(def initial-state 0)
(def transition
  (fn [current-state input]
    (+ current-state input)))
\end{lstlisting}
Given such a function, and an \texttt{initial-state}, it's simple to turn
\rad{} into the specified state-machine:
\begin{lstlisting}
(def state (ref initial-state))
(def new-eval
  (fn [e]
    (write-ref state
               (transition (read-ref state) e))))

(write-ref eval-ref new-eval)
\end{lstlisting}

However, in so doing we have lost the ability to modify \texttt{transition} in
any way.

So as to allow for such modifications, we define a state-machine which runs other
state-machines while also accepting meta-commands for operations such as voting
for a new transition function.

Inputs are partitioned into \emph{basic inputs}:
\[
\mathtt{(basic} \ i),
\]
where $i$ are values accepted by \texttt{transition}, and \emph{meta-inputs}:
\[
\mathtt{(meta} \ c),
\]
where the $c$ are messages participants use to coordinate in choosing a new
transition function. All basic-inputs are handled by \texttt{transition}
to update the current state. Examples of meta-inputs are:
\begin{align*}
  (\text{\texttt{new-transition-function}} \ f \ u)\\
  (\text{\texttt{vote-agree}} \ uid \ s)\\
  (\text{\texttt{vote-disagree}} \ uid \ s)
\end{align*}
where
\begin{itemize}
\item $f$ is a new transition function,
\item $u$ is a function used to upgrade the state, if it now has a new format,
\item $uid$ is a user identifier,
\item $s$ is a cryptographic signature to validate the vote.
\end{itemize}
The resulting function \texttt{run-state-machine} could also take in other
parameters, for example to instantiate different voting processes, etc. In fact,
one could conceivably include a command for upgrading the whole function which
handles meta-level commands.

\subsection{Pull request chain}

As an example we will consider a chain used to manage the state of
a \emph{pull request} (PR, a request to merge a git branch into another). In this
case the state being maintained by the chains is composed of:
\begin{itemize}
  \item the discussion about the PR,
  \item which users have accepted or rejected the change,
  \item the final `result', one of \texttt{:accepted}, \texttt{:rejected} or
    \texttt{:undecided}.
\end{itemize}

We'll assume that we have a few functions defined already:
\begin{itemize}
  \item A function \texttt{result-from-votes} which takes a dict from user IDs
    to votes (one of \texttt{:accept} or \texttt{:reject}) and returns a result.
  \item A library function \texttt{modify-dict} takes three arguments: a
    dictionary, a key and a function. It returns a new version of the dictionary
    where the value at the key has been updated according to the function.
\end{itemize}

\begin{lstlisting}
;; In the initial state there are no comments,
;; no votes, and the PR is still undecided.
(def initial-state
  {:comments []
   :votes    {}
   :result   :undecided})

;; After any input we update the result.
(def update-result
  (fn [state]
    (def new-result
         (result-from-votes
           (lookup :votes state)))
    (insert :result new-result state)))

;; Adds a comment to the list of comments.
(def add-comment
  (fn [state comment]
    (modify-dict
      :comments
      (fn [cs] (append cs c))
      state)))

;; Updates a user's vote.
(def update-votes
  (fn [state user-vote]
    (def user-id (lookup :user-id user-vote))
    (def vote (lookup :vote user-vote))
    (modify-dict
      :votes
      (fn [votes] (insert user-id vote votes))
      state)))

;; The state transition function.      
(def transition
  (fn [state input]
    (def command (nth 0 input))
    (def arg (nth 1 input))
    (update-result
      (cond
        (eq? command 'vote)
        (update-votes state arg)
        (eq? command 'comment)
        (add-comment state arg)))))
\end{lstlisting}

As in the currency example, in a real setting we would also add cryptographic
signatures to comments and votes in order to verify that they are only submitted
by the stated user.
