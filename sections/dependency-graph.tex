\section{The Dependency Graph}
\subsection{The Dependency Game}

% TODO: "As an individual entity, you are not a party to the flow of value
% in the economy."

To incentivize the building of a dependency graph for open-source, we create a
mechanism which rewards participants for identifying dependencies correctly and
punishes them for acting maliciously.

There are two types of players in the game: \emph{proposers} and
\emph{verifiers}.  The former propose a dependency between two projects which
they believe to be valid, and place a security deposit $G_{\delta}$ which they
are prepared to lose if they are lying. The later verify proposals by casting a
vote which can have either of two values: $1$ (\emph{true}), for proposals
believed to be valid, and $0$ (\emph{false}), for proposals considered
invalid. Verifiers who manage to agree on $G_d$ without communicating
are each rewarded with an amount of tokens $\delta$.

% TODO: Problem: how to stop validators from communicating.
% TODO: Automation for proposers and validators.

A dependency game $G$ is initiated by a proposer $G_p$ and progresses in a
sequence of rounds $G_r \equiv r_1 \dots r_n$, where $n \geqslant 1$. Each
round $r$ attempts to resolve the dependency $G_d$ (the ``proposal''), by
either marking it \emph{true} or \emph{false}, and involves two verifiers,
their votes $r_{v_1}$ and $r_{v_2}$, and their deposits $r_{\delta_1}$ and
$r_{\delta_2}$. When $r_{v_1} \equiv r_{v_2}$, the game ends.  Formally, we
have:

\begin{align*}
    G      & \equiv \tuple{G_d, G_p, G_{\delta}, G_r} \\
    G_r    & \equiv \{r_1, \dots, r_n\} \\
    r      & \equiv \tuple{r_{v_1}, r_{v_2}, r_{\delta_1}, r_{\delta_2}} \\
    r_v    & \in \{0, 1\} \\
\end{align*}

% We recall that a dependency is represented by the tuple $\tuple{D_a, D_b, D_e}$
% described in~\S~\ref{dependencies}.

The game starts with an initial state $G_0$, where $G_{0_r} = \varnothing$.
The first round $r_1$ is played as such, where $r \equiv r_1$:

\begin{enumerate}
    \item Two players with deposits $r_{\delta}$ are selected at random to
        become verifiers, and presented with $G_d$. Their identities are not
        revealed to one another.
    \item The players each cast a vote $r_{v_1}$ and $r_{v_2}$ signaling
        whether they believe $G_d$ to be \emph{true} or \emph{false}. The votes
        are not revealed until both players have cast their vote.
\end{enumerate}

Now, the system is in either of three states:

\begin{enumerate}
    \item[(a)] $r_{v_1} \equiv r_{v_2}$, \qquad $r_{v_1} = 1 \; \wedge \; r_{v_2} = 1$
    \item[(b)] $r_{v_1} \equiv r_{v_2}$, \qquad $r_{v_1} = 0 \; \wedge \; r_{v_2} = 0$
    \item[(c)] $r_{v_1} \not\equiv r_{v_2}$
\end{enumerate}

If we find ourselves in state (a), all deposits are returned to the
participants, and $r_{V_1}$ and $r_{V_2}$ are rewarded.  In the case of (b),
the proposer loses their deposit $G_{\delta}$, and it is split between
$r_{V_1}$ and $r_{V_2}$.  Finally, in (c), the verifiers have voted
differently, and there is no way to resolve the round.  Therefore we proceed
with the following:

\begin{enumerate}
    \item A new round is initiated with the same proposal $G_d$, and two new
        verifiers $r_{V_3}$ and $r_{V_4}$, which are picked at random:
        \begin{equation*}
            r_2 \equiv \tuple{r_{v_3}, r_{v_4}, r_{\delta_3}, r_{\delta_4}}
        \end{equation*}
    \item Again, we have three outcomes:
        \smallskip
        \begin{enumerate}
            \item[(a)] $r_{v_3} \equiv r_{v_4}$, \qquad $r_{v_3} = 1 \; \wedge \; r_{v_4} = 1$
            \item[(b)] $r_{v_3} \equiv r_{v_4}$, \qquad $r_{v_3} = 0 \; \wedge \; r_{v_4} = 0$
            \item[(c)] $r_{v_3} \not\equiv r_{v_4}$
        \end{enumerate}
        \smallskip
        In both outcomes (a) and (b), $r_{v_3}$ and $r_{v_4}$ settle the
        previous round $r_1$, by forming a $\frac{3}{1}$ majority vote for
        either $1$ or $0$. This causes the minority voters to lose their
        deposits to the majority. Note that in each round where agreement is
        not reached, we have an additional $2\delta$ at stake.

        In the case of (c), we proceed the same way as in $r_1$, initiating a
        new round with new verifiers.  Each additional round increases the
        total value at stake. Since additional rounds do not cost the system
        more, there is no limit to the number of rounds.
\end{enumerate}

\begin{table}[hbt]
    \caption{Outcomes for $G$ when $G_d$ was voted \emph{invalid}}
    \begin{tabular}{lccrrrc}
    \toprule
        \hspace   & \multicolumn{2}{c}{Vote} & \multicolumn{3}{c}{Reward} & Cost                    \\
    \midrule
        \hspace   & $r_{v_i}$  & $r_{v_j}$   & $r_{v_i}$ & $r_{v_j}$ & $G_p$    & $r_{v_i}+r_{v_j}$ \\
    \addlinespace[0.5em]
        $r_{n-2}$ & $1$        & $0$         & $\delta$  & $-\delta$ &          & $0$               \\
        $r_{n-1}$ & $0$        & $1$         & $-\delta$ & $\delta$  &          & $0$               \\
        $r_{n}$   & $1$        & $1$         & $\delta$  & $\delta$  & $\delta$ & $3\delta$         \\
    \bottomrule
    \end{tabular}
\end{table}

\begin{table}[hbt]
    \caption{Outcomes for $G$ when $G_d$ was voted \emph{valid}}
    \begin{tabular}{lccrrrc}
    \toprule
        \hspace   & \multicolumn{2}{c}{Vote} & \multicolumn{3}{c}{Reward} & Cost                     \\
    \midrule
        \hspace   & $r_{v_i}$  & $r_{v_j}$   & $r_{v_i}$ & $r_{v_j}$ & $G_p$     & $r_{v_i}+r_{v_j}$ \\
    \addlinespace[0.5em]
        $r_{n-2}$ & $1$        & $0$         & $-\delta$ & $\delta$  &           & $0$               \\
        $r_{n-1}$ & $0$        & $1$         & $\delta$  & $-\delta$ &           & $0$               \\
        $r_{n}$   & $0$        & $0$         & $\delta$  & $\delta$  & $-\delta$ & $\delta$          \\
    \bottomrule
    \end{tabular}
\end{table}

