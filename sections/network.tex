\section{The Oscoin Network}

\subsection{Entities}

\noindent Several distinct entities comprise the network.  Entities which
transact on the network are called \emph{participants}. The following
participants exist on the \oscoin{} network.

\addvspace{1em}

\begin{description}
    \item[User] A user is any person or bot active on \oscoin{}.  Activity
        includes browsing and submitting code, reading and commenting on
        issues, and any other interaction allowed on \oscoin{}'s network.
    \item[Maintainer] A user is considered a \emph{maintainer} if they
        are responsible for the maintenance, organization and general health of
        one or more \emph{projects}.
    \item[Contributor] A user is considered a \emph{contributor} if they
        perform a \emph{public transaction} on the network which is considered
        to have value by one or more \emph{maintainers}. This includes but is
        not limited to committing code, opening and closing issues, commenting
        and voting.
    \item[Operator] A user is considered an \emph{operator} if they are involved
        in the core operation of the network. This includes any user operating
        physical hardware on which the \oscoin{} services run, and typically
        involves transaction processing and validation. Operators in \oscoin{}
        are equivalent to \emph{miners} in Bitcoin. The term \emph{validator}
        is often used interchangeably with \emph{operator}.
\end{description}

\addvspace{1em}
\noindent We can define the \emph{network}, $\mathcal{N}$ as the set of all active
participants. Beyond active participants, the following social entities
exist:

\begin{description}
    \item[Projects] A project is a collection of source code, issues,
        documentation, and other objects commonly found in version
        controlled repositories. Notably, several users contribute to a
        project.
    \item[Organizations] At the highest level we have the organization: a
        collection of projects governed by users. Example organizations
        include large software foundations like the Linux Foundation,
        decentralized autonomous organizations (DAOs) such as Dash, and
        individuals managing their own personal projects. Organizations are
        the unit of governance in \oscoin{}.
\end{description}
\addvspace{1em}

And finally, tying all participants and entities together,

\begin{description}
    \item[The Ledger, $\mathcal{L}$] acts as a global source of truth for all
        participants in the network and holds the totally ordered set of all
        \emph{valid} transactions having taken place. These transactions are
        organized in \emph{epochs}, representing periods of time. If $t$ is a
        an epoch, then $\mathcal{L}_t$ is the state of the ledger at epoch $t$.
        All participants in the network have access to the latest state
        $\mathcal{L}_t$.
\end{description}

